\documentclass[a4paper,oneside,10pt,DIV12,headsepline,footexclude,headexclude]{scrartcl}
\begin{document}
    \begin{section}{ASP}
        \begin{itemize}
            \item Definieren Sie den Begriff eines Amwer Sets eines logischen Progrnrnmes f' .
Wie gehen Sie vor, um die Answer Sets eines gegebenen Programms zu berechnen: welche "!
Schritte sind in welcher Reihenfolge durchzuführen?  
            \item Was versteht man unter Abduktion und einem abduktiven Diagnoseproblem?  
            \item Was versteht man unter einer konsistenzbasierten Diagnose?
            Besteht aus Hypothese, einer Theorie, einem Set von Beobachtungen.
            \item Was ist ein klassisches Modell eines Programms P?
        \end{itemize}
    \paragraph {Multiple Choice}
        \begin{enumerate}
            \item Regeln in einem Programm zur konsitenzbasierten Diagnose muessen grundiert sein. (Falsch)
            \item Das leere Programm hat kein Answer Set. (Falsch)
            \item Es gibt ein normales logisches Programm, welches ein Answer Set besitzt das sowohl
            ein Atom $a$ als auch dessen Negation $\neg a$ enthaelt. (Falsch)
            \item Ein Answer Set eines normalen grundierten Programms P kann kein Atom enthalten,
            dessen Praedikatensymbol nicht im Kopf einer Regel von P vorkommt. ()
        \end{enumerate}    
    \rule{\textwidth}{1pt}        
    \begin{enumerate}
            \item
            Wenn $M_1$ ein Answer Set eines Programms $P_1$ ist, und $M_2$ ein
            Answer Set eines Programms $P_2$, dann ist $M_1 \cup M_2$ ein Answer Set von $P_1 \cup P_2$. (Falsch)
            \item Wenn M ein minimales Modell eines Programms P ist, dann ist M ein Answer Set von P. (Falsch)
            \item Abduktive Diagnosen sind ein schwaecheres Konzept als consitency-based diagnosis. (Falsch)
            \item Jede Teilmenge von $\{a,b,c\}$ auszer der leeren Menge ist ein Answer Set von $P = \{a \vee b \vee c :-\}$ (Falsch)
        \end{enumerate}    
    \rule{\textwidth}{1pt}        
    \begin{enumerate}
        \item Ein Answer Set eines normalen Programms P kann nicht-grundierte Atome enthalten (Falsch)
        \item Das Programm $P = \{a \vee b :-, a \vee c :-\}$ hat die Answer Sets $\{a\}, \{b,c\}, \{a,b\}, \{a,c\}$ (Falsch)
        \item Fuer jedes $n \geq 1$ gibt ein disjunktives logisches Programm, in welchem $\Theta (n)$ Atome vorkommen
        , welches jedoch mindestens $2^n$ Answer Sets besitzt. (Richtig)
        \item Regeln in einem Programm zur abduktiven Diagnose duerfen disjunktiv sein (Falsch)
        \item Jedes klassische Modell eines Programms P ist auch ein Answer Set von P (Falsch).
   \end{enumerate}

    \rule{\textwidth}{1pt}        
    
    \begin{enumerate}
        \item Wenn $M_1$ ein Answer Set eines PRogrammes $P_1$ ist, und $M_2$ ein Answer Set eines Programms $P_2$, dann ist $M_1 \cup M_2$ ein Answer Set von $P_1 \cup P_2$. (Falsch)
        \item Wenn M ein minimales Modell eines Programms P ist dann ist M ein Answer Set von P (Falsch)
        \item Abduktive Diagnosen sind ein schwaecheres Konzept als consitency-based Diagnosen (Falsch)
        \item Jede Teilmenge von $\{a, b, c\}$ auszer der leeren Menge ist ein Answer Set von $P = \{a \vee b \vee c:-\}$ ()
    \end{enumerate}

    \rule{\textwidth}{1pt}

    \begin{enumerate}
        \item Es gibt grundierte, normale Answer-Set Programme, die keine Answer Sets besitzen. ()
        \item Ein Answer Set eines normalen Programms P kann kein Atom enthalten dass nicht im Kopf einer Regel von P vorkommt. (Wahr)
        \item Regeln in einem Programm zur konsistenzbasierten Diagnose duerfen nicht disjunktiv sein. (Wahr)
        \item Jede Teilmenge von $\{a, b, c\}$ auszer der leeren Menge ist ein Answer Set von $P = \{a \vee b \vee c \leftarrow \}$ (Falsch)
    \end{enumerate}

    \rule{\textwidth}{1pt}

    \begin{enumerate}
    \item Jedes Hornprogramm hat ein klassisches Modell  (Wahr????)
    \item Das Programm $P = \{a \vee b :-, a \vee c :-, :- a \vee b\}$ ist ein disjunktives Logisches Programm? (Falsch)
    \item Falls ein Programm ein klassisches Modell hat, so besitzt es auch ein Answer Set, jedoch sind nicht notwendigerweise alle klassische Modelle Answer Sets. (Falsch)
    \item Constraints fuegen keine Ausdrucksstaerke hinzu, sie koennen auf normale Regeln reduziert werden. (Wahr)
    \item Sei $P$ ein Programm mit starker Negation und $P'$ ein Programm welches aus P entsteht in dem wir alle literale der Form $\neg p$ uniform durch ein neues Atom $q_{\neg p}$ ersetzen.
    Falls $P$ kein Answer Set besitzt, so auch $P'$. ()
    \end{enumerate}

    \rule{\textwidth}{1pt}
    \begin{enumerate}
    \item Ein Answer Set eines normalen Programms P kann kein Atom enthalten dessen Praedikatensymbol nicht im Kopf einer Regel von P vorkommt. (Richtig)
    \item Regeln in einem Programm zur konsistenzbasierten Diagnose duerfen Disjunktiv sein. ()
    \item Jede Teilmenge von $\{a,b,c\}$ auszer der Leeren Menge ist ein Answer Set von $ P = \{a \vee b \vee c :-\}$ (Falsch)
    \item Es existieren Interpretationen $M_1, M_2$ und Programme $P_1, P_2$ sodass $M_1$ ein Answerset von $P_1, M_2$, ein Answerset von $P_2$, und $M_1 \cup M_2$ ein Answer Set von $P_1 \cup P_2$ ist. (Richtig)
    \item Jedes klassische Modell eines Programms P ist auch ein Answer Set von P. (Falsch)
    \end{enumerate}
    \rule{\textwidth}{1pt}
    \begin{enumerate}
    \item Wenn $M_1$ ein Answer Set eines Programms $P_1$ ist und $M_2$ ein Answer Set eines Programms $P_2$ dann ist $M_1 \cup M_2$ ein Answer Set von $P_1 \cup P_2$. (Falsch)
    \item Wenn M ein minimales Modell eines Programms P ist dann ist M ein Answer Set von P. (Falsch)
    \item Abduktive Diagnosen sind ein schwaecheres Konzept als consitency based Diagnosis. (Falsch)
    \item Jede Teilmenge von $\{a,b,c\}$ auszer der Leeren Menge ist ein Answer Set von $ P = \{a \vee b \vee c :-\}$ (Falsch)
    \end{enumerate}

    \end{section}

    \rule{\textwidth}{1pt}

\end{document}
